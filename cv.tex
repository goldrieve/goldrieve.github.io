\documentclass[a4paper,11pt]{article}

%-----------------------------------------------------------
% FONT
%-----------------------------------------------------------
\usepackage[default]{helvet} 
\renewcommand{\familydefault}{\sfdefault}

%-----------------------------------------------------------
% PACKAGES
%-----------------------------------------------------------
\usepackage{url}
\usepackage{parskip}
\usepackage{graphicx}
\usepackage[usenames,dvipsnames]{xcolor}
\usepackage[scale=0.9]{geometry}
\usepackage{tabularx}
\usepackage{enumitem}
\usepackage{titlesec}
\usepackage{multicol}
\usepackage{fontawesome5}
\usepackage[unicode]{hyperref}
\usepackage[most]{tcolorbox}

% link styling
\definecolor{linkcolour}{rgb}{0.0,0.2,0.0}
\hypersetup{colorlinks, urlcolor=linkcolour, linkcolor=linkcolour}

% job environments
\newenvironment{joblong}[2]{
  \begin{tabularx}{\linewidth}{@{}l X r@{}}
  \textbf{#1} & & #2 \\[2pt]
  \end{tabularx}
  \begin{itemize}[leftmargin=1.2em, itemsep=3pt, label=-]
}{
  \end{itemize}
}

\newenvironment{jobshort}[2]{
  \begin{tabularx}{\linewidth}{@{}l X r@{}}
  \textbf{#1} & & #2 \\[2pt]
  \end{tabularx}
}{}

% section style
\titleformat{\section}{\large\bfseries\uppercase}{}{0em}{}[\titlerule]
\titlespacing{\section}{0pt}{8pt}{6pt}

% Define a skill box style
\newtcbox{\skillbox}[1][]{%
  on line,
  colframe=black!40,
  colback=black!5,
  boxrule=0.4pt,
  arc=3pt,
  boxsep=2pt,
  left=4pt,
  right=4pt,
  top=2pt,
  bottom=2pt,
  #1}

%-----------------------------------------------------------
\begin{document}
\pagestyle{empty}

%-----------------------------------------------------------
% HEADER
%-----------------------------------------------------------

\begin{center}
    {\Huge \textbf{Guy Oldrieve, PhD}} \\[6pt]
    \href{mailto:oldrieve2@hotmail.com}{\faEnvelope\ oldrieve2@hotmail.com}  \ $|$ \ 
    \href{https://github.com/goldrieve}{\faGithub\ goldrieve}  \ $|$ \ 
    \href{https://linkedin.com/in/guy-oldrieve}{\faLinkedin\ guy-oldrieve}
    \end{center}

%-----------------------------------------------------------
\begin{multicols}{2}
\subsection*{Summary}
Over the past six years, I have applied my passion for bioinformatics to projects focused on Hosts, Pathogens and Global Health. I specialise in developing user-friendly and efficient pipelines to analyse and manage large-scale multi-omic projects.

\columnbreak

\subsection*{Skills}
\skillbox{Python} \skillbox{R} \skillbox{bash} \skillbox{Nextflow} \skillbox{nf-test} \skillbox{GitHub+Actions} \skillbox{AWS/HPC} \skillbox{ML}
\end{multicols}

%-----------------------------------------------------------
\section{Experience}

\begin{joblong}{Bioinformatician | Animal and Plant Health Agency}{2025 - Present}
\item Developed (btb-forestry) and improved (btb-seq) nf-test CI capacity for ISO-17025 certified Nextflow pipelines used to process sequencing data and perform phylogenetic analysis of bovine TB isolates
\item Created btb-sub, a nextflow pipeline which automates the submission of routine TB sequencing data to ENA
\end{joblong}

\begin{joblong}{Postdoctoral Research Associate | The University of Edinburgh}{2023 - 2025}
\item Created vsgseq2, an efficient Nextflow pipeline that processes cDNA amplicon data to anlayse the expression of complex gene families used for antigenic variation (GitHub: vsgseq2)
    \begin{itemize}[leftmargin=2em, itemsep=2pt, label=$\circ$]
        \item vsgseq2 utilises nf-test based unit/integration testing and Docker to ease distribution
        \item Utilised vsgseq2 to analyse hundreds of longitudinal in vivo infection samples.
        \item Generated high-quality genome assemblies from PacBio, ONT and HI-C sequencing to understand the genomic background of expression analysis
    \end{itemize}
\item Developed a real-time diagnostic tool to detect outbreaks of T. brucei in the field using minION-based amplicon sequencing
    \begin{itemize}[leftmargin=2em, itemsep=2pt, label=$\circ$]
        \item Created an app which allows submission of data via a GUI
        \item Performed the wet lab based amplicon generation and sequencing
        \item Simultaneously supervised two students in wet lab and bioinformatics
    \end{itemize}
\item Delivered team training in HPC use and managed HPC environments, ensuring secure data handling and streamlined analysis workflows
\item Co-authored the first profile of the Trypanosoma brucei cell cycle with single-cell RNAseq
\item Council member for the British Society for Parasitology, promoting the adoption of emerging technologies through web interface tutorials
\item Reviewed articles for journals such as Trends in Parasitology, PLOS NTD, Frontiers in Cellular and Infection Microbiology and Parasitology Research
\end{joblong}

\begin{joblong}{PhD in Hosts, Pathogens, and Global Health | The University of Edinburgh}{2018 - 2023}
\item Developed a Snakemake pipeline to investigate the phylogenetic association and mechanisms behind life cycle simplification in Trypanosoma brucei. Identified specific mutations which limit a cell's ability to receive oligopeptide-based quorum sensing signals
\item Confirmed my in silico predictions through genetic manipulations with CRISPR/Cas9
\item Sequenced and assembled the genome of a neglected parasite remotely using a MinION, emphasising independent research and problem-solving during pandemic-enforced absence from the lab
\item Presented findings to scientific and general audiences at international conferences
\item Supervised and mentored a student researcher, teaching them the required skills to assemble a genome and perform large-scale comparative genomics, leading to a co-authored publication
\end{joblong}

\begin{joblong}{MSc in Hosts, Pathogens and Global Health | The University of Edinburgh}{2018 - 2019}
\item Identified the relationship between 100's of bacterial isolates with machine learning and data visualisation in R. 
\item Created an R-based data mining approach to compare 1,000s of unstructured datasets to identify biomarkers of severe malaria.
\end{joblong}

\begin{joblong}{Research Assistant | Cardiff University}{2013 - 2018}
\item Utilised machine learning tools (scikit-learn) to classify phylogenetic relationships of transcriptome data.
\end{joblong}

%-----------------------------------------------------------
\section{Education}
\begin{itemize}[leftmargin=1.2em, itemsep=3pt, label=-]
\item \begin{tabularx}{\linewidth}{@{}X r@{}}
PhD in Hosts, Pathogens, and Global Health at University of Edinburgh \textbf{(Ker Memorial Prize)} & 2018 - 2023 \\
\end{tabularx}
\item \begin{tabularx}{\linewidth}{@{}X r@{}}
MSc in Hosts, Pathogens and Global Health at University of Edinburgh \textbf{(Merit)} & 2018 - 2019 \\
\end{tabularx}
\item \begin{tabularx}{\linewidth}{@{}X r@{}}
Integrated Master's in Biology (MBiol) at Cardiff University \textbf{(First-class)} & 2013 - 2018 \\
\end{tabularx}
\end{itemize}

%-----------------------------------------------------------
\section{Publications}
\begin{itemize}[leftmargin=1.2em, itemsep=3pt, label=-]
\item \textbf{Oldrieve G.}, et al. (2025). vsgseq2: an updated pipeline for analysis of the diversity and abundance of population-wide Trypanosoma brucei VSG expression. In Prep.: Wellcome Open Research.
\item Vancaester E., \textbf{Oldrieve G.}, et al. (2025) Ghosts of symbionts past: The hidden history of the dynamic association between filarial nematodes and their Wolbachia endosymbionts. Accepted: G3.
\item \textbf{Oldrieve G.}, et al. (2024) Mechanisms of life cycle simplification in African trypanosomes. Nature Communications.
\item Briggs E., Marques C., \textbf{Oldrieve G.}, et al. (2023). Profiling the Trypanosoma brucei cell cycle using single-cell transcriptomics. eLife.
\item \textbf{Oldrieve G.}, et al. (2022). The genomic basis of host and vector specificity in non-pathogenic trypanosomatids. Biology Open.
\item \textbf{Oldrieve G.}, et al. (2021). Monomorphic Trypanozoon: towards reconciling phylogeny and pathologies. Microbial Genomics.
\end{itemize}

%-----------------------------------------------------------
\section{Awards}
\begin{itemize}[leftmargin=1.2em, itemsep=3pt, label=-]
\item \begin{tabularx}{\linewidth}{@{}X r@{}}
\textbf{Ker Memorial Prize} for the most outstanding PhD thesis in infectious diseases from across the University of Edinburgh & (2024) \\
\end{tabularx}
\item \begin{tabularx}{\linewidth}{@{}X r@{}}
\textbf{Best poster} at the Edinburgh Infectious Diseases symposium & (2020) \\
\end{tabularx}
\item \begin{tabularx}{\linewidth}{@{}X r@{}}
\textbf{Best performance} Integrated master's in biology & (2018) \\
\end{tabularx}
\end{itemize}

%-----------------------------------------------------------
\section{Interests}
\begin{itemize}[leftmargin=1.2em, itemsep=3pt, label=-]
\item \begin{tabularx}{\linewidth}{@{}X r@{}}
Endurance Events: Ljubljana, Loch Ness and Edinburgh marathon & (2023-Present) \\
\end{tabularx}
\item \begin{tabularx}{\linewidth}{@{}X r@{}}
Captain of Lismore RFC: Led a diverse group of over 50 players & (2020-2023) \\
\end{tabularx}
\end{itemize}

\end{document}